\documentclass[12pt]{scrartcl}

\setlength{\parindent}{0pt}
\setlength{\parskip}{.25cm}

\usepackage{graphicx}

\usepackage{xcolor}

\definecolor{darkred}{rgb}{0.5,0,0}
\definecolor{darkgreen}{rgb}{0,0.5,0}
\usepackage{hyperref}
\hypersetup{
  letterpaper,
  colorlinks,
  linkcolor=red,
  citecolor=darkgreen,
  menucolor=darkred,
  urlcolor=blue,
  pdfpagemode=none,
  pdftitle={Introduction To Git},
  pdfauthor={Christopher M. Bourke},
  pdfcreator={$ $Id: cv-us.tex,v 1.28 2009/01/01 00:00:00 cbourke Exp $ $},
  pdfsubject={PhD Thesis},
  pdfkeywords={}
}

\definecolor{MyDarkBlue}{rgb}{0,0.08,0.45}
\definecolor{MyDarkRed}{rgb}{0.45,0.08,0}
\definecolor{MyDarkGreen}{rgb}{0.08,0.45,0.08}

\definecolor{mintedBackground}{rgb}{0.95,0.95,0.95}
\definecolor{mintedInlineBackground}{rgb}{.90,.90,1}

%\usepackage{newfloat}
\usepackage[newfloat=true]{minted}
\setminted{mathescape,
               linenos,
               autogobble,
               frame=none,
               framesep=2mm,
               framerule=0.4pt,
               %label=foo,
               xleftmargin=2em,
               xrightmargin=0em,
               startinline=true,  %PHP only, allow it to omit the PHP Tags *** with this option, variables using dollar sign in comments are treated as latex math
               numbersep=10pt, %gap between line numbers and start of line
               style=default, %syntax highlighting style, default is "default"
               			    %gallery: http://help.farbox.com/pygments.html
			    	    %list available: pygmentize -L styles
               bgcolor=mintedBackground} %prevents breaking across pages
               
\setmintedinline{bgcolor={mintedBackground}}
\setminted[text]{bgcolor={mintedBackground},linenos=false,autogobble,xleftmargin=1em}
%\setminted[php]{bgcolor=mintedBackgroundPHP} %startinline=True}
\SetupFloatingEnvironment{listing}{name=Code Sample}
\SetupFloatingEnvironment{listing}{listname=List of Code Samples}

\title{CSCE 155 - C}
\subtitle{Lab 07 - Arrays \& Dynamic Memory}
\author{Dr.\ Chris Bourke}
\date{~}

\begin{document}

\maketitle

\section*{Prior to Lab}

Before attending this lab:
\begin{enumerate}
  \item Read and familiarize yourself with this handout.
  \item Review the following free textbook resources:
	\begin{itemize}
  	  \item \url{http://www.cs.cf.ac.uk/Dave/C/node7.html} 
	  \item \url{http://en.wikibooks.org/wiki/C_Programming/Arrays}
	  \item \url{http://www.cs.cf.ac.uk/Dave/C/node11.html}
	  \item \url{http://en.wikibooks.org/wiki/C_Programming/Memory_management}
	\end{itemize}
\end{enumerate}

\section*{Peer Programming Pair-Up}

To encourage collaboration and a team environment, labs will be
structured in a \emph{pair programming} setup.  At the start of
each lab, you will be randomly paired up with another student 
(conflicts such as absences will be dealt with by the lab instructor).
One of you will be designated the \emph{driver} and the other
the \emph{navigator}.  

The navigator will be responsible for reading the instructions and
telling the driver what to do next.  The driver will be in charge of the
keyboard and workstation.  Both driver and navigator are responsible
for suggesting fixes and solutions together.  Neither the navigator
nor the driver is ``in charge.''  Beyond your immediate pairing, you
are encouraged to help and interact and with other pairs in the lab.

Each week you should alternate: if you were a driver last week, 
be a navigator next, etc.  Resolve any issues (you were both drivers
last week) within your pair.  Ask the lab instructor to resolve issues
only when you cannot come to a consensus.  

Because of the peer programming setup of labs, it is absolutely 
essential that you complete any pre-lab activities and familiarize
yourself with the handouts prior to coming to lab.  Failure to do
so will negatively impact your ability to collaborate and work with 
others which may mean that you will not be able to complete the
lab.  

\section{Lab Objectives \& Topics}
At the end of this lab you should be familiar with the following
\begin{itemize}
  \item Understand what memory leaks are, how they're caused, 
  	and how to prevent them
  \item Declare and fill an array with values
  \item Access an array's values individually and iterate over the entire array
  \item Pass an array as a parameter to a function
  \item Understand the relationship between arrays and pointers
\end{itemize}

\section{Background}

Arrays hold collections of variable values of a certain type 
(\mintinline{c}{int} or \mintinline{c}{double} for example).
Static arrays are arrays whose size is specified at compile 
time and specified by hardcoding the size when declared.  
For example:

\mintinline{c}{int array[100];}

This declares an integer array that holds 100 integers (in 
addition, static arrays are allocated on the program stack 
while dynamic arrays are allocated on the program heap, 
but we will not go into those details here).  Once we have 
an array, we can access individual elements in the array by 
specifying an index:

\begin{minted}{c}
array[0] = 5;
printf("the last element is %d\n", array[99]);
\end{minted}

You should take care that you do not access elements beyond 
the range of the array's indices.  Indices start at 0 (first element) 
and end one less than the size of the array.  Attempts to access 
elements outside this range results in undefined behavior and 
may result in a segmentation fault or other such run time error.

Unfortunately, static arrays are quite limited and for most applications, 
we will not know how big an array needs to be at compile time.  
Instead, we typically need an array whose size is not known until 
runtime.  Dynamic arrays provide such functionality.

Dynamic arrays are arrays whose size is not specified at compile 
time.  Rather, we write code such that at run time, a chunk of 
memory is dynamically allocated to accommodate the space we 
need to hold a certain number of variable values.  C provides a 
family of functions to dynamically allocate memory in the standard 
C library (\mintinline{text}{stdlib.h}): \mintinline{c}{malloc()}, 
\mintinline{c}{calloc()} and \mintinline{c}{realloc()}.  

In this lab, we will focus on malloc (memory allocation), which takes 
exactly one argument: the number of bytes you wish to allocate in 
memory.  To make our code portable, we do not hardcode the number 
of bytes each type of variable requires.  Instead, we use the 
\mintinline{c}{sizeof} operator to determine how many bytes each 
type takes.  Finally, \mintinline{c}{malloc} returns a generic void 
pointer (\mintinline{c}{void *}) because as far as \mintinline{c}{malloc} 
is concerned, it is just allocating memory, it doesn't care what you 
intend to use it for.  Therefore, it is best practice to explicitly type 
case the void pointer into the type of pointer that we intend to use 
it as.  Altogether, we have:

\mintinline{c}{int *arr = (int *) malloc(sizeof(int) * 100);}

This creates an array that is holds 100 integers.  Once allocated, 
we can treat the array like any other static array with indices 
starting at 0 just as before.

Memory is a resource and when using dynamic memory, memory 
management becomes an issue.  Memory management means 
that we responsibly ``clean up'' after ourselves.  Once the dynamic 
memory that was allocated is no longer needed, we need to tell 
the operating system that we're done with it and that it should be 
made available to be reused by our program or allow other 
programs to use it.  We release memory by using the \mintinline{c}{free()} 
function and passing it a pointer to our memory chunk:

\mintinline{c}{free(arr);}

Failure to release unneeded memory may result in a memory leak 
whereby a program continually allocates memory but never releases 
it, becoming a resource hog.  Eventually, the program uses so much 
memory that it slows  to a halt or is terminated as the operating 
system is unable or unwilling to allocate more resources to it.

\section{Activities}

Clone the repository from GitHub containing the code for this lab by using 
the following URL: \url{https://github.com/cbourke/CSCE155-C-Lab07}.

\subsection{Observing a Memory Leak}

In this exercise, we'll observe a memory leak in action.

\subsubsection*{Instructions}
\begin{enumerate}
  \item Change to the \mintinline{text}{memoryLeak} directory
  \item Build the executable using the \mintinline{text}{make} command.  
  	Make is a utility that reads a makefile--a specification for source file 
	dependencies and the process by which they are built.  The makefile 
	for this project has been provided for you.
  \item A program named \mintinline{text}{memLeak} will be created in your 
	directory.  You can configure and run it as follows:
	
	\mintinline{text}{./memLeak 1000000 10}
	
	which will run the program and create 10 random arrays each of size
	1 million integer elements.  It then sorts the array and reports
	the median element.  The program will execute rather quickly so it
	will be difficult to observe the memory leak.

  \item To observe the consequences of the leak, run the program with
    the following configuration:	

	\mintinline{text}{./memLeak 10000000 20 > /dev/null &}
	
	This will run the program with more arrays that are much larger, 
	slowing it down.  It will also suppress the output and run the
	program in the background so you can continue operating on the
	command line.

  \item Run the command \mintinline{text}{top -u login} where \mintinline{text}{login}
	is replaced with your cse login.  This will open up a list of all of the 
	processes currently being executed by you.  
	
  \item Observe the column labeled ``RES''.  This column will tell you 
    the current amount of memory being used by a specific process.  
    It should take about 2 minutes to execute.  Type \mintinline{text}{q}
    to quit out of the \mintinline{text}{top} program.
\end{enumerate}

\subsection{Diagnosing \& Fixing a Memory Leak}

In this activity, you'll use a dynamic analysis tool called 
Valgrind (\url{http://valgrind.org/}) to diagnose the memory leak
in this program.  Valgrind is a \emph{dynamic analysis} tool that
can monitor a program and analyze the resources it is using and
call attention to errors that may occur at runtime.  

\subsubsection*{Instructions}
\begin{enumerate}
  \item Run the Valgrind tool using the following command:
  
\mintinline{text}{valgrind --leak-check=full --show-leak-kinds=all ./memLeak 1000000 10}
   This starts Valgrind and runs your program within it.  With the above 
   parameters, it shouldn't take as long as before.
  \item Once it is done, it should produce a report that includes a total number of
  bytes that was lost due to the memory leak as well as where the memory
  was originally allocated (though not where it was lost).  
  \item Using this report as a guide, fix the memory leak and answer 
  the questions on your worksheet
\end{enumerate}
  
\subsection{Static Arrays}

Navigate back to the statistics directory.  Several files have been 
provided for you.  The \mintinline{text}{stats.c} and \mintinline{text}{stats.h} 
file define several utility functions to manipulate arrays and to 
compute statistics on those arrays.  The \mintinline{text}{statsMain.c} 
file contains a main driver program.  Recall that you will compile 
these files using:

\begin{minted}{text}
gcc -c stats.c
gcc -o runStats stats.o statsMain.c
\end{minted}

In this activity, you will implement and use several functions that use 
arrays as parameters.  When passed as a parameter to a function, 
the function also needs to be told how large the array is; typically an 
integer size variable is passed with any array parameter.  You will also 
declare and populate a static array to test your functions.

\subsubsection*{Instructions}

\begin{enumerate}
  \item A function, \mintinline{c}{readInArray} has been provided for you 
	that takes an array and its size and prompts the user to enter values 
	to populate the array.  However, there is an error in the function: 
	examine the \mintinline{c}{scanf} call and determine what is wrong 
	and fix it before proceeding.
  \item Using the other functions as a reference, implement the 
	\mintinline{c}{getMin}, \mintinline{c}{getMax}, and \mintinline{c}{getMean} 
	functions.  You will need to provide the correct function signature 
	in both the header file and source file.
  \item In the \mintinline{text}{statsMain.c} declare a static array that is 
	``large enough'' as indicated by the other clues in the code.  Use your 
	declared static array as an argument to \mintinline{c}{readInArray} 
	and the other function calls.
  \item Run your program and demonstrate it to a lab instructor.
\end{enumerate}

\subsection{Dynamic Arrays}

In this activity you will modify the code in \mintinline{text}{statsMain.c} to 
use a dynamic array instead of a static array.

\subsubsection*{Instructions}

\begin{enumerate}
  \item Alter your static array declaration to be an integer pointer and add 
	code that calls malloc to initialize the appropriate amount of memory.
  \item Alter any other code as necessary to remove the ``large enough'' 
	restriction that we had in the previous activity.
  \item Run your program again and demonstrate it to a lab instructor
\end{enumerate}
	
\subsubsection*{Follow Up}

Demonstrate your program using a ``large'' array.  Obviously prompting 
for and entering a lot of numbers is tedious.  So instead, alter your code 
again to instead take advantage of the \mintinline{c}{createRandomArray} 
function provided for you.  Demonstrate it to a lab instructor and have 
them sign your lab worksheet. 

\section{Advanced Activity (Optional)}

Read more about the capabilities of Valgrind (\url{http://valgrind.org/}, 
\url{http://www.cs.yale.edu/homes/aspnes/pinewiki/C(2f)Debugging.html#Valgrind}
and how to use it.  Think of other situations that this tool may have come in
handy and try using it for the next bug that you encounter.  


\end{document}
